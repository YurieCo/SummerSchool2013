\documentclass[t]{beamer}
% If you don't have access to the Imperial College beamer theme, then
% commenting out the following line will produce a generic version of the
% slides. 
\usetheme{iclpt}

\usepackage{color,listings}
\usepackage{pstricks, pst-node}
\usepackage{pgfpages,xspace,array}

\newcommand{\doc}[1]{\psshadowbox[framearc=0]{#1}}
\newcommand{\program}[1]{\psframebox[framearc=.2]{#1}}

\usepackage{graphicx,stmaryrd,cancel}
\usepackage{bibentry}
\usepackage{natbib}
\usepackage{hyperref}
\usepackage{amscd}
\usepackage{booktabs}
\bibliographystyle{elsarticle-harv}
\usepackage{amsmath}

\renewcommand{\vec}[1]{\boldsymbol{#1}}

\lstset{frame=single}

\expandafter\ifx\csname natexlab\endcsname\relax\def\natexlab#1{#1}\fi
\expandafter\ifx\csname url\endcsname\relax
  \def\url#1{\texttt{#1}}\fi
\expandafter\ifx\csname urlprefix\endcsname\relax\def\urlprefix{URL }\fi

\definecolor{icdarkblue}{RGB}{23,18,134}
\definecolor{icdarkgreen}{RGB}{0,130,0}

%\setbeameroption{show notes on second screen}

\author[David A. Ham]{Dr David~A.~Ham} 
\date{24 May 2013}

\title{Verification and validation of simulation software}
                  
\institute[Imperial College London]{
  Department of Computing, Imperial College London\\
  Grantham Institute for Climate Change, Imperial College London
david.ham@imperial.ac.uk}

\begin{document}
\nobibliography{bibliography}

\begin{frame}{}
  \vfill{}

  \centering

  \Large\color{icdarkblue}\inserttitle\\
  %\normalsize\insertsubtitle\\[3ex]
  \small\color{black}\insertauthor\\[3ex]
  \footnotesize\insertinstitute

  \vfill{}

  With much material from \bibentry{Farrell2011}

\end{frame}

\begin{frame}{How do we know whether to believe the output of a model?}
  \vfill{}

  \pause
  This is fundamentally a mixed question of mathematics, software
  engineering, and application science. 
  \vfill{}

  \pause
  It's also a critical question which all computational scientists have to
  answer if they expect other scientists or society at large to take note of
  their results.
  \vfill{}
  

\end{frame}

\begin{frame}{A brief trip back to philosophy of science 101}
  
  \begin{itemize}
  \item Mathematical results are \emph{proven}. In other words, if the
    assumptions of a theorem are satisfied, then the result \emph{always}\
    follows. There is no possibility of another result.
  \item Science proceeds by \emph{hypotheses}. These can never be proven in
    the mathematical sense, but they can be \emph{falsified}\ through a
    contrary observation\footnote{\bibentry{popper1959},\\\bibentry{howden1976}}.
  \end{itemize}
  
  So what hypotheses does our software present?

\end{frame}

\begin{frame}{Verification and validation operations}

  \includegraphics[width=\textwidth]{output/verification.pdf}
  
\end{frame}

\begin{frame}{Verification and validation operations}
  
  \begin{itemize}
  \item Only the discretisation of the PDE is usually a mathematical
    operation which can be proven.
  \item Verification of software by mathematical methods is practiced in
    some fields of computing, but simulation code is typically far beyond
    their scope.
  \item The relationship between the continuous model and the physical
    system cannot be directly measured, so we must go through \emph{all}\
    the other steps. 
  \end{itemize}
  
\end{frame}

\begin{frame}{If it's not tested, it's broken.}
  
  \begin{description}
  \item[Unit tests] Tests of correct behaviour applied to the smallest
    possible unit of code.
  \item[Analytic solutions] Very strong tests of correctness of numerics and
    implementation, if you can get one!
  \item[Method of manufactured solutions] A mechanism for generating
    analytic solutions.
  \item[Regression tests] Tests against a previously computed result. Only a
    test of change, not of correctness.
  \item[Third party solution] Tests against another model. Better than
    regression tests, but only as good as the other model.
  \item[Comparisons to ``real'' data] Essential in validating a model, but of
    limited use in verification.
  \end{description}

\end{frame}

\begin{frame}{Some PDE testing theory}

  First, know what convergence means for your numerics. Numerical schemes
  for PDEs usually have convergence behaviour given by an expression such
  as:
  \begin{equation}
    E_h=O(h^n)
  \end{equation}
  Where $E_h$ is the error in the numerical solution, $h$ is a measure of the
  mesh spacing and $n$ is the order of convergence of the method.
  
  It is important to understand both sides of this expression.

  \hfill{}

  Testing convergence behaviour of numerical schemes is a very sensitive
  verification test, since almost any error in the numerics will degrade the
  convergence order.\footnote{At least for schemes of greater than first order.}

\end{frame}

\begin{frame}{Definition of error}
  
  Finite volume and finite element schemes converge in (at least) the space
  $L^2$. So the definition:
  \begin{equation}
    E_h = \left(\int_\Omega(\hat{S}-S)^2\mathrm{d}V\right)^{\frac{1}{2}}
  \end{equation}
  is often appropriate. Here $\Omega$ is the solution domain, $\hat{S}$ is
  the numerical solution and $S$ is the exact solution.
  \vfill{}
  
  Importantly, this is \emph{not}\ the same as pointwise evaluation of the
  solution. 

\end{frame}

\begin{frame}{L2-norm for low-order finite volume}
  
  For low-order finite volume, the $L^2$-norm is approximated by summing over
  control volumes:
  \begin{equation}
    E_h \approx \left(\sum_{v\in\Omega}\left(\mathrm{vol}(v)\left(\hat{S}_v - S(\vec{x}_v)\right)\right)^2\right)^{\frac{1}{2}}
  \end{equation}
  where $\vec{x}_v$ is the location of the control point. This creates an
  $O(h^2)$ error in the integration of the analytic solution, so is only
  acceptable for schemes of second order and below.
  \vfill{}

  \textbf{General rule:} remember to always ensure that the error you make
  measuring the error is smaller than the error itself!

\end{frame}

\begin{frame}{Evaluating convergence rate}
  Let's return to:
  \begin{equation}
    E=O(h^n)
  \end{equation}
  and which means:
  \begin{equation}
    \lim_{h\rightarrow 0} E_h\leq Ch^n
  \end{equation}
  For some $C$. \pause Further, if $n$ is the optimal convergence rate of the
  scheme, then for \emph{sufficiently small} $h$:
  \begin{equation}
    E_h\approx Ch^n    
  \end{equation}

\end{frame}

\begin{frame}{Evaluating convergence rate}

  Suppose we run the simulation on two meshes with typical (small) mesh spacing
  $h_1$ and $h_2$:
  \begin{eqnarray} 
    E_{h_1} & \approx & Ch_1^{n}\textrm{,} \\ E_{h_2} &
    \approx & Ch_2^{n}\textrm{,} 
\end{eqnarray}
\pause
\begin{equation}
  \frac{E_{h_1}}{E_{h_2}} \approx \left(\frac{h_1}{h_2}\right)^n
\end{equation}
\pause
\begin{equation}
  n \approx \log_{h_1/h_2}\left(\frac{E_{h_1}}{E_{h_2}}\right)
\end{equation}

\end{frame}

\begin{frame}{Method of manufactured solutions}

  MMS is a ridiculously simple idea. 
  
\end{frame}

\begin{frame}{Test driven development}
  
  Core commandments of the cult:
  

\end{frame}



\begin{frame}{Software tools for verification}
  
\end{frame}

\begin{frame}{Coverage}
  
\end{frame}

\end{document}
